%  -*- coding: utf-8 -*-
% !TeX program = pdflatex

\documentclass[7pt]{article}

\usepackage[T1]{fontenc}
\usepackage{lmodern}
\setlength{\parindent}{1cm}
\usepackage[bottom=3cm,top=2cm,margin=2.25cm]{geometry}



\begin{document}
  \title{Basics of Integrals}
  \date{\today}
  \maketitle
  \tableofcontents
  \centerline{\line(1,0){450}}

  \section[Explanation]{Explaining Notation}

  An integral, simply represents the sum of the area of infinitesimally thin
  rectangles. The \textit{thinness} or width of the rectangle is denoted as $dx$
  which represents an infinitesimally small increment in the $x$ direction.
  Thus the area of each rectangle is $dx \cdot y$ as $y$ of course, represents the
  height of the rectangle.\\

  The way we used to estimate the area under a graph, was making loads of small
  rectangles under the curve and add together all the areas; this is imprecise.
  In order to get better results we need to make the rectangles smaller, thus
  needing more of them.\\
  
  \subsection{Expressing Yourself}
  By using integrals we can imagine infinitesimally thin rectangles being added up
  to give a much more precise answer for the area under the curve. The notation
  for integrals is just a big `S' that looks like this: `$\int_{a}^{b}$'. You may
  notice the `a' and `b', this can be read as `The integral of [equation of line] form $a$ to $b$',
  with $a$ and $b$ being $x$ co\"{o}rdinates, limiting the area of the graph we're actually
  calculating.\\\\
  Now, say we have the equation for the curve:
  \[ f(x) = 4x^2 - \frac{1}{2}x + 3 \]
  and we want to find the area under that curve from $x = 4$ to $x = 12$
  \[ \int_{4}^{12} f(x)\;dx \]
  \centerline{or, with $f(x)$ expanded}
  \[ \int_{4}^{12} 4x^2 - \frac{1}{2}x + 3\;dx \]\\
  \subsection{Semantics}
  Let's break this down: ignore the integral sign for now, and lets just focus
  on one rectangle. We have our equation for the height (the $y$ co\"ordinate, $f(x)$),
  now we need to multiply that by some infinitesimally small nudge in $x$ to form our
  infinitesimally thin rectangle, giving us: $dx$.\\

  $d$ is just notation for an infinitesimally small nudge in a certain direction.
  This gives us the general equation for any rectangle under the curve: $y \cdot dx$.
  Moving on to the integral sign, which simply just stands for ``sum'',
  because we are summing area of \textbf{all} the rectangles between $a$ and $b$
  ($4$ and $12$) thusly giving us the total area under the curve $f(x)$ 
  where $4 \leq x \leq 12$.

\end{document}
