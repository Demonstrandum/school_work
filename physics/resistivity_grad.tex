%  -*- coding: utf-8 -*-
% !TEX program = pdflatex

\documentclass{article}
\usepackage{circuitikz}
\usepackage{varwidth}
\usepackage[margin=3.35cm]{geometry}

\usepackage{pgfplots,pgfplotstable}
\pgfplotsset{/pgf/number format/use comma,compat=newest}

\begin{document}

\title{Physics - Resistance and Resistivity over a variable length wire}
\author{Samuel F. D. Knutsen}
\date{Compiled: \today}
\maketitle

\tableofcontents
\clearpage

\section[Aim]{The Aim of the Test}
\subsection{Evaluating Resistivity for a Wire}
The ultimate aim of the test was to study the relationship between the length of a wire, and how that impacts resistance by measuring the current in, and the voltage across the variable length wire. With the constant cross-sectional area, we can then calculate the resisting power of the material (the resistivity). To calculating resistivity we can use the equation:

\[ \varrho = R \cdot \frac{A}{l} \]

\begin{center}
  \begin{varwidth}{\textwidth}
    Where:\\`$\varrho$' (`rho') is the resistivity,\\
    `$R$' is the resistance,\\
    `$A$' is the cross-sectional area, and\\
    `$l$'\ \ is the length of the wire.
  \end{varwidth}
\end{center}

\section[Circuit]{Circuit Diagram}
\subsection{Diagram}
\begin{circuitikz} \draw
(0,0) to[battery] (0,4)
      to[ammeter] (4,4) -- (4,0)
      to[R=$R$] (0,0)
(0.5,0) -- (0.5,-2)
      to[voltmeter] (3.5,-2) -- (3.5,0)
;
\end{circuitikz}
\subsection{Explanation}
A battery supplies current to the circuit, the ammeter measures the current and is in series with the circuit. A wire shown as $R$ on the diagram sits between the negative and positives nodes of the battery. 
A voltmeter is in parallel across the wire, measuring the potential difference across the wire. \\

With these two measurements we can calculate the resistance using our previous knowledge of the formula: \[V=IR\] \centerline{and rearrange to give us:} \[R=\frac{V}{I}\]

\clearpage

\section[Results]{Values and Data}
\subsection{Uncertainty}
The percentage uncertainty of a micrometer is about $0.01\%$, the multimeter was marked with a $0.03\%$ uncertainty and the finest markings on our ruler was in in millimetres, giving an uncertainty of 0.5 mm (0.0005 m), so we'll divide 0.0005 by the length of our measurement to obtain our percentage uncertainty.

\begin{center}
\begin{tabular}{ |c|c|c| }
 \hline
 Length (m) & Resistance ($\Omega$) & Uncertainty (\%) \\
 \hline
 0.1 & 0.19 & $\pm\ 1/2$ \\ 
 0.2 & 0.37 & $\pm\ 1/4$ \\  
 0.3 & 0.53 & $\pm\ 1/6$ \\
 0.4 & 0.70 & $\pm\ 1/8$ \\
 0.5 & 0.88 & $\pm\ 1/10$ \\
 0.6 & 1.01 & $\pm\ 1/12$ \\
 0.7 & 1.21 & $\pm\ 1/14$ \\
 0.8 & 1.45 & $\pm\ 1/16$ \\
 0.9 & 1.62 & $\pm\ 1/18$ \\
 1.0 & 1.79 & $\pm\ 1/20$ \\
\hline
\end{tabular}
\\ \vspace{2mm} Where our cross-sectional area is: $0.0003$ $m^2$
\end{center}

\subsection{Graph}
See attached graph with uncertainties/error-bars plotted.

\subsection{Data Analysis}
As is evident and as suspected, while the length of the wire increases so does the resistance. We know that as the length increases, the current it's capable of diminishes.\\

By taking the line of best fit of the graph, where we are assuming no uncertainty, we end up at a gradient at roughly $1.78\dot{3}$, gained from taking 
\[ \nabla = \frac{\overline{\Delta R}}{\Delta l} \]

\section[Evaluation]{Evaluating Results}
\subsection{Equations}
We remind ourselves of the basic proportionality for resistance where resistivity is the constant of proportionality ($\varrho$):
\[ R \propto \frac{l}{A} \]
\begin{center}{Which we can expand to an equation and rearrange for
\\the resistivity (the constant of proportionality denoted as `$\varrho$'):}
\[ \varrho = R \cdot \frac{A}{l} \]
\end{center}
\clearpage

\subsection{Computing Resistivity}
We can now begin to compute resistivity using the previous equation; the values of the resistivity over the variable length wire have been tabulated below:
\begin{center}
\begin{tabular}{ |c|c|c|c| }
 \hline
 Length (m) & Resistance ($\Omega$) & Resistivity ($\Omega m$) & Uncertainty (\%) \\
 \hline
 0.1 & 0.19 & $\varrho = 0.19 \cdot \frac{0.0003}{0.1} = 0.000570$ & $\pm\ 1/2$ \\ 
 0.2 & 0.37 & 0.000555 & $\pm\ 1/4$ \\  
 0.3 & 0.53 & 0.000530 & $\pm\ 1/6$ \\
 0.4 & 0.70 & 0.000525 & $\pm\ 1/8$ \\
 0.5 & 0.88 & 0.000528 & $\pm\ 1/10$ \\
 0.6 & 1.01 & 0.000505 & $\pm\ 1/12$ \\
 0.7 & 1.21 & 0.000519 & $\pm\ 1/14$ \\
 0.8 & 1.45 & 0.000544 & $\pm\ 1/16$ \\
 0.9 & 1.62 & 0.000540 & $\pm\ 1/18$ \\
 1.0 & 1.79 & 0.000537 & $\pm\ 1/20$ \\
\hline
\end{tabular}
\\ \vspace{2mm}
($\%$ uncertainty, resistance and length increments remain the same as from the last table.)
\\ \vspace{5mm}

\end{center}

As hoped/expected the values for the resistivity are somewhat constant, and well within the constraints of the percentage uncertainty. The variable in the experiment is the length of the wire and thusly the resistance will vary with it as well, this demonstrates that the resistivity of the wire will not change by the length of the material.\\

So far this satisfies our definition of resistivity as: ``A measure of the resisting power of a specific \emph{material} to the flow of an electric current.'', notice how we are taking the measurement of a property of the \emph{material}, which naturally should not be impacted by the length of the object of that material.

\subsection{Gradient and Resistivity}
By doing our manual calculations for each length of wire to find our constants of resistivity which are more or less constant across all the different length of wires .\\

Some of the values are slightly out of range, so we'll take the average of all the calculated resistivity values to get an average of: 0.000535 $\Omega$ metres.\\

This has been one method of calculating the resistivity. An important aspect to remember is that the area (`$A$') as the coefficient to the gradient of the resistance-length graph is actually equal to the resistivity of the material.
To demonstrate, let's take our gradient `$\nabla$', and use our cross-sectional area (`$A$') as a coefficient to this gradient, to ultimately get the resistivity with our obtained equation:

\[ \varrho = A \cdot \nabla \]

which we can substitute our original values in to (the gradient (`$\nabla$') as $1.78\dot{3}$ and the cross-sectional area (`$A$') as 0.0003 $m^2$), to get:

\[ \varrho = 0.0003 \cdot 1.78\dot{3} = 0.00053499 \]

our obtained value of 0.00053499 rounds to 0.000535 (6 d.p.) which is the very same value we obtained from calculating the average of all of our individual resistivity results, this, further verifies our methods and results.

\end{document}
